\documentclass[acmtog]{acmart}
\usepackage{graphicx}
\usepackage{subfigure}
\usepackage{natbib}
\usepackage{listings}
\usepackage{bm}
\usepackage{amsmath}

\definecolor{blve}{rgb}{0.3372549 , 0.61176471, 0.83921569}
\definecolor{gr33n}{rgb}{0.29019608, 0.7372549, 0.64705882}
\makeatletter
\lst@InstallKeywords k{class}{classstyle}\slshape{classstyle}{}ld
\makeatother
\lstset{language=C++,
	basicstyle=\ttfamily,
	keywordstyle=\color{blve}\ttfamily,
	stringstyle=\color{red}\ttfamily,
	commentstyle=\color{magenta}\ttfamily,
	morecomment=[l][\color{magenta}]{\#},
	classstyle = \bfseries\color{gr33n}, 
	tabsize=2
}
\lstset{basicstyle=\ttfamily}

% Title portion
\title{Final Project:\\ {Liquid Simulation by SPH Method}} 

\author{Name:\quad JiaYing Du + AIBO Hu  \\ student number:\ 2021533037 + ?
\\email:\quad dujy2@shanghaitech.edu.cn + ?}

% Document starts
\begin{document}
\maketitle

\vspace*{2 ex}

\section{Introduction}
\begin{itemize}
\item ... has been down.
\item ...
\item The surface rendering is done.
\end{itemize}
\section{Implementation Details}
\item In the surface rendering part, I adopted the marchcube method for its simplicity and wide usage.\\
The marchcube method is a method described below:\\
When You are processing a cloud of particles you can compute the density of them like in the steps above. Therefore, 
you can assume at a sample point there is a particle, you can compute the density of the hypothetical particle.\\
We figure out the possible space where liquid may appear and we divide the space with mesh cube. We can get sample points lying on 
every cross point of the mesh cube. We compute the density for each point. We can get the condition of each small grid.\\
Having got the condition for each small grid. We set a threshold density and then binarize the values. We can design a map mapping each condition of small grid ($2^8$) to 
different surface conditions, i.e. different vertices and different faces. Because the cubes are adjacent we have to design the map carefully. 
Therefore I use the universially recogized marching cube map.\\
Then we can get the vertices and faces of the surface. Drawing them out we get the surface rendering outcome.
\section{Results}
% pictures should be in

\end{document}
